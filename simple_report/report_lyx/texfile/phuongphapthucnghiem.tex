\chapter{Phương pháp thực nghiệm}
\label{ch:thucnghiem}


\section{Một số loại detector}

Trong chương \ref{ch:cosolythuyet} đã trình bày các nguyên lý ghi nhận bức xạ, trong phần này sẽ trình bày các phương pháp thực nghiệm để ghi nhận các tia $\gamma$.

Một detector ghi nhận $\gamma$ được thiết lập từ các detector nhỏ hơn trong đó mỗi detector nhỏ sẽ gồm các khối Germanium được đặt trong một lớp bảo vệ. Các khối Germanium có thể được phân chia thành các vùng khác nhau  và mỗi vùng này sẽ được kết nối với một điện cực riêng. Trong phần này sẽ giới thiệu sơ lược về bốn loại detector khác nhau: EXOGAM, EUROBALL, JUROGAM và AGATA\cite{bib_Bazzaco, bib_Simpson} với các thông số chính như trong Bảng \ref{table:detector}. Mỗi loại detector đều có những ưu điểm riêng, tuy nhiên detector AGATA cho phép ghi nhận các tia bức xạ  $\gamma$ với độ phân giải tốt nhất. Để có thể tái lập năng lượng của các tia bức xạ $\gamma$, thuật toán addback và tracking đã được sử dụng.

\begin{table}[!h]
\caption{So sánh một số detector}
\centering
\begin{tabular}{|c|c|c|c|}
\hline 
Loại Detector & Hệ thống điện tử & Thuật toán áp dụng & sử dụng Trigger \\ 
\hline 
EUROBALL & analog & addback & có \\ 
\hline 
EXOGAM & analog & addback & có \\ 
\hline 
JUROGAM & analog + digital & không & không \\ 
\hline 
AGATA & digital & tracking $\gamma$ & trigger+triggerless \\ 
\hline 
\end{tabular} 
\label{table:detector}
\end{table}

Các kết quả thực nghiệm sẽ được trình bày trong Chương \ref{ch:ketquaphantich}.
 


