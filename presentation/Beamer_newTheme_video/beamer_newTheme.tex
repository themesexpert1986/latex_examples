\documentclass[10pt, t]{beamer}

\usepackage[T1, T2A,T2B,T2C,T5]{fontenc}
\usepackage[utf8]{inputenc}
\usepackage[english, french, russian, main = vietnamese]{babel}

\usepackage{amssymb, amsthm}
\usepackage[intlimits]{mathtools}

\usepackage{graphicx}

\usepackage{tikz}
\usetikzlibrary{trees, backgrounds, arrows, shapes, shadows, calc}



\usepackage{movie15}

\usepackage[timeinterval = 1]{tdclock}



\usepackage{pgfpages}
\usepackage{handoutWithNotes}
\pgfpagesuselayout{4 on 1 with notes}[a4paper,border shrink=5mm]


% các thông số của văn bản
\renewcommand{\baselinestretch}{1.3}%*\label{line:lvtotnghiep_definition_renewcommand}*)
\setlength{\parindent}{1.27 cm}
\setlength{\parskip}{6pt}%*\label{line:lvtotnghiep_definition_setlength}*)

% xoá các heading trên các trang trắng
\newcommand{\clearHeading}{%*\label{line:lvtotnghiep_definition_clearHeading}*)
	\clearpage	
	\pagestyle{empty}     
	\cleardoublepage
	\pagestyle{fancyplain}
}

% định dạng trang
\pagestyle{fancyplain} %*\label{line:lvtotnghiep_definition_fancyplain}*)
\renewcommand{\chaptermark}[1]{%
	\markboth{\sl \chaptername~\thechapter.~#1}{}}
\renewcommand{\sectionmark}[1]{%
	\markright{\sl \thesection.~#1}}
  
	\fancyhf{}
	\lhead[\fancyplain{}{\leftmark}]{}
	\rhead[]{\fancyplain{}{\rightmark}}
	\lfoot[\thepage]{}
	\rfoot[]{\thepage}%*\label{line:lvtotnghiep_definition_rfoot}*)
	
	
% các định nghĩa sử dụng trong trang bìa
\makeatletter %*\label{line:lvtotnghiep_definition_makeatletter}*)
\newcommand{\@khoa}{Khoa gì?}
\newcommand{\khoa}[1]{\gdef\@khoa{#1}}
\newcommand{\@nganh}{Ngành học?}
\newcommand{\nganh}[1]{\gdef\@nganh{#1}}

\newcommand{\@chuongtrinhdaotao}{Chương trình gì?}
\newcommand{\chuongtrinhdaotao}[1]{\gdef\@chuongtrinhdaotao{#1}}

\newcommand{\@nam}{Năm ?}
\newcommand{\nam}[1]{\gdef\@nam{#1}}

\newcommand{\@thayhuongdan}{Tên thầy hướng dẫn ?}
\newcommand{\thayhuongdan}[1]{\gdef\@thayhuongdan{#1}} %*\label{line:lvtotnghiep_definition_thayhuongdan}*)

% trang bìa
\renewcommand{\maketitle}{ %*\label{line:lvtotnghiep_definition_maketitle}*)
\begin{titlepage}%

\newgeometry{top=2cm, bottom=2cm, left=2cm, right=2cm}%*\label{line:lvtotnghiep_definition_maketitle_geo}*)
	
\AddToShipoutPicture*{%	
\AtTextCenter{%
	\makebox(0,0)[c]{
	\begin{tikzpicture}
 	\draw[line width = 4pt](0,0) rectangle (\textwidth+2*0.3cm, 
 		\textheight+2*0.3cm);
	\end{tikzpicture}
 	}
}
}

\begin{center}
	
{\fontsize{13}{15.6}\selectfont 
 ĐẠI HỌC QUỐC GIA HÀ NỘI\\
 ĐẠI HỌC KHOA HỌC TỰ NHIÊN\\
 \textbf{KHOA: \MakeUppercase\@khoa}} 

	\vfill
	\textbf{\@author}
	\vfill
 	{\fontsize{18}{21.6}\selectfont \textbf{\MakeUppercase\@title}}
 	\vfill
 	
	 Khóa luận tốt nghiệp đại học hệ chính quy
	\vspace{\baselineskip}
 	
	Ngành: \@nganh\\
	(Chương trình đào tạo: \@chuongtrinhdaotao)
    
    \vfill
    \vfill
    
     \textbf{Hà Nội - \@nam}
  
  \end{center}
 \end{titlepage}%
} %*\label{line:lvtotnghiep_definition_maketitle_end}*)


% trang bìa phụ
\newcommand{\maketitleTrangphu}{%*\label{line:lvtotnghiep_definition_maketitleTrangphu}*)
\begin{titlepage}%

\AddToShipoutPicture*{%	
\AtTextCenter{%
	\makebox(0,0)[c]{
	\begin{tikzpicture}
	\draw[line width = 4pt](0,0)rectangle(\textwidth+2*0.3cm,
		\textheight+2*0.3cm);
 	\end{tikzpicture}
 	}
 }
}


\begin{center}
		
 {\fontsize{13}{15.6}\selectfont% 
 ĐẠI HỌC QUỐC GIA HÀ NỘI\\
 ĐẠI HỌC KHOA HỌC TỰ NHIÊN\\
 \textbf{KHOA: \MakeUppercase\@khoa}}

	\vfill
	\textbf{\@author}
	\vfill
 	{\fontsize{18}{21.6}\selectfont \textbf{\MakeUppercase\@title}}
	 \vfill
	Khóa luận tốt nghiệp đại học hệ chính quy
	\vspace{\baselineskip}
 	Ngành: \@nganh\\
    (Chương trình đào tạo: \@chuongtrinhdaotao)
    
    \vfill
	\textbf{Cán bộ hướng dẫn: \@thayhuongdan}
    \vfill
     
    \textbf{Hà Nội - \@nam}
    
\end{center}
\end{titlepage}%
}%*\label{line:lvtotnghiep_definition_maketitleTrangphu_end}*)

\makeatother

%\includeonlyframes{ghinhan}

%%%%%%%%%%%%%%%%%%%%%%%%%%%%%%
\title[Cấu trúc hạt nhân 156Gd]{Ứng dụng phổ $\gamma$ trong nghiên cứu cấu trúc hạt nhân 156Gd}
\author[D.Q. Tuyền]{Đoàn Quang Tuyền}
\institute[IPNL]{Viện nghiên cứu hạt nhân Lyon, Pháp \\  4 Rue Enrico Fermi, 69622 Villeurbanne, France}
\date[Zakopane, 2008]{Zakopane, Ba lan, 1-7/09/2008} 

\begin{document}


\logo{\includegraphics[height=0.14\paperheight]{Logo/Logo-ipnl.jpg}}

\begin{frame}
\logo{}
\insertlogo{
\includegraphics[height=0.14\paperheight]{Logo/tetanuclogo}
\hfill
\includegraphics[height=0.14\paperheight]{Logo/Agalogo}
}
\titlepage
\initclock
\end{frame}

\logo{}

\begin{frame}{Các nội dung chính}
\tableofcontents
\end{frame}

\section{Tổng Quan}

\subsection{Các phổ $\gamma$}

\begin{frame}[label = ghinhan]{Ghi nhận $\gamma$}

Các thông số chính của detector:

\begin{itemize}

\item Hiệu suất ghi nhận: $\varepsilon_{p} = \dfrac{N_\text{(ghi nhận)}}{N_\text{(phát ra)}}$

\item Tỷ số P/T: $P/T = \dfrac{N_\text{(ghi nhận)}(E_{\gamma} = 1000 keV)}{N_\text{(ghi nhận)}}$

\item Độ phân giải $\Delta E$

\end{itemize}

Độ phân giải của phổ $\gamma$ phụ thuộc vào năng lượng của tia tới và góc tán xạ (góc mở của detector).

\end{frame}


\begin{frame}[label = chamluongtu]{Các chấm lượng tử}

Chuyển động Brown của các chấm lượng tử: 

\begin{center}

\includemovie[controls,toolbar,inline=true,mouse=true,autopause,%
autoplay,continue,autoclose,autostop,attach=true,
playerid=MSFT_WindowsMediaPlayer
]{0.5\textwidth}{0.5\textheight}{video/video.avi}

\end{center}

\end{frame}


%
%\includemovie[controls,toolbar,inline=true,mouse=true,autopause,%
%autoplay,continue,autoclose,autostop,attach=true,%
%%playerid=RNWK_RealPlayer%
%%playerid=AAPL_QuickTime%
%%playerid=MSFT_WindowsMediaPlayer%
%%playerid=MACR_FlashPlayer%
%%playerid=ADBE_MCI%
%]{0.5\textwidth}{0.5\textwidth}{video/video.avi}


\subsection{Cấu trúc hạt nhân 156Gd}


%%%%%%%%%%%%%
\section{Các phương pháp thực nghiệm}
\subsection{Các loại detector}



\subsection{Xử lý số liệu}

%%%%%%%%%%%%%%%
\section{Các kết quả chính}
\begin{frame}{Số liệu tiết diện}
\end{frame}

\subsection{Các chuyển dịch}
\begin{frame}{Sai số}
\begin{block}{Chào}
Thông tin về phương trình
\end{block}
\end{frame}

%%%%%%%%%%%%
\section{Kết luận}
\begin{frame}{Kết luận}
\end{frame}

\end{document}