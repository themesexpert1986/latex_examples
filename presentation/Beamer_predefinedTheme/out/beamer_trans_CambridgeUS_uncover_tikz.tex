\documentclass[11pt, hyperref={unicode}]{beamer}

\usetheme{CambridgeUS}

\usepackage[T1, T2A,T2B,T2C,T5]{fontenc}
\usepackage[utf8]{inputenc}
\usepackage[english, french, russian, main = vietnamese]{babel}

\usepackage{amssymb, amsthm}
\usepackage[intlimits]{mathtools}
\usepackage{graphicx}

\usepackage{tikz}
\usetikzlibrary{trees, backgrounds, arrows, shapes, calc}

\includeonlyframes{nguyentu}


\begin{document}

\begin{frame}
	\titlepage
\end{frame}

\begin{frame}{Các nội dung chính}
	\tableofcontents
\end{frame}

\section{Tổng Quan}

\begin{frame}[label = nguyentu]{Cấu tạo nguyên tử}

Cấu tạo nguyên tử:
\begin{tikzpicture}

	\node[draw](A) at (0,0){Atom};
	\node[draw](B) at (15:4){Nucleus};
	\node[draw](C) at (-15:5){electron};
	
	\draw[->](A)|-(B);
	\draw[->](A)-|(C);
	
	\uncover<2->{\draw[->](A)edge[bend left](B);}
	\uncover<2->{\draw[->](A)edge[bend right](C);}
	\uncover<3->{\draw[->](B)edge[out=270, in=180](C);}
	
\end{tikzpicture}

\end{frame}

...
...

\end{document}