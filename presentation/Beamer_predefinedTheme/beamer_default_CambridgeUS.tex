\documentclass[11pt, hyperref={unicode}]{beamer}
\usetheme{CambridgeUS}

\usepackage[T1, T2A,T2B,T2C,T5]{fontenc}
\usepackage[utf8]{inputenc}
\usepackage[english, french, russian, main = vietnamese]{babel}

\usepackage{amssymb, amsthm}
\usepackage[intlimits]{mathtools}

\usepackage{graphicx}

\title[Cấu trúc hạt nhân 156Gd]{Ứng dụng phổ $\gamma$ trong nghiên cứu cấu trúc hạt nhân 156Gd}
\author[D.Q. Tuyền]{Đoàn Quang Tuyền}
\institute[IPNL]{Viện nghiên cứu hạt nhân Lyon, Pháp \\  4 Rue Enrico Fermi, 69622 Villeurbanne, France}
\date[Zakopane, 2008]{Zakopane, Ba lan, 1-7/09/2008} 

\AtBeginSection[] {
	\begin{frame}{Các nội dung chính}
 		\tableofcontents[currentsection]
	\end{frame}
	\addtocounter{framenumber}{-1}
}

%\includeonlyframes{ghinhan}

\begin{document}

\begin{frame}
	\titlepage
\end{frame}

\begin{frame}{Các nội dung chính}
	\tableofcontents
\end{frame}


\section{Tổng Quan}

\begin{frame}[label = ghinhan]{Ghi nhận $\gamma$}

Các thông số chính của detector:

\begin{itemize}

\item<2-> Hiệu suất ghi nhận: $\varepsilon_{p} = \dfrac{N_\text{(ghi nhận)}}{N_\text{(phát ra)}}$

\item<3-> Tỷ số P/T: $P/T = \dfrac{N_\text{(ghi nhận)}(E_{\gamma} = 1000 keV)}{N_\text{(ghi nhận)}}$

\item<4-> Độ phân giải $\Delta E$

\end{itemize}

\uncover<5->{Độ phân giải của phổ $\gamma$ phụ thuộc vào năng lượng của tia tới và góc tán xạ (góc mở của detector).}

\end{frame}



%%%%%%%%%%%%%
\section{Các phương pháp thực nghiệm}
\begin{frame}{Các loại detector}
\end{frame}

\begin{frame}{Các chương trình xử lý số liệu}
\end{frame}


%%%%%%%%%%%%%%%
\section{Các kết quả chính}
\begin{frame}{Số liệu tiết diện}
\end{frame}

\begin{frame}{Sai số}
\end{frame}

%%%%%%%%%%%%
\section{Kết luận}
\begin{frame}{Kết luận}
\end{frame}

\end{document}