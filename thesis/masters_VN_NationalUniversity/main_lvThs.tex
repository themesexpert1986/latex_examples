\documentclass[a4paper,14pt, twoside, openright]{extreport}

\usepackage{times}
\usepackage{mathptmx}

\usepackage[utf8]{inputenc} 
\usepackage[T5]{fontenc} 
\usepackage[vietnamese]{babel}
 
\usepackage[bottom=2.5cm, top=2.5cm, inner=3.5cm, outer=2cm]{geometry}

\usepackage{amssymb, amsthm}
\usepackage[intlimits]{mathtools}
\usepackage{graphicx}

\usepackage[colorlinks, breaklinks, pdfencoding=unicode]{hyperref}

\usepackage[intoc]{nomencl}%*\label{line:lvthacsi_main_nomencl}*)

\usepackage{indentfirst}

\usepackage{fancyhdr}

\usepackage[defernumbers, backend=biber, style=numeric, citestyle = numeric-comp, sorting = nty]{biblatex} 
	\addbibresource{bibtex/biblio_vn.bib} 
	\addbibresource{bibtex/biblio_en.bib}

\usepackage{tikz}
\usetikzlibrary{calc}
\usepackage{eso-pic} 

\usepackage[toc]{appendix}

\usepackage{tocloft}%*\label{line:lvthacsi_main_tocloft}*)

	% các thông số của văn bản
\renewcommand{\baselinestretch}{1.3}%*\label{line:lvtotnghiep_definition_renewcommand}*)
\setlength{\parindent}{1.27 cm}
\setlength{\parskip}{6pt}%*\label{line:lvtotnghiep_definition_setlength}*)

% xoá các heading trên các trang trắng
\newcommand{\clearHeading}{%*\label{line:lvtotnghiep_definition_clearHeading}*)
	\clearpage	
	\pagestyle{empty}     
	\cleardoublepage
	\pagestyle{fancyplain}
}

% định dạng trang
\pagestyle{fancyplain} %*\label{line:lvtotnghiep_definition_fancyplain}*)
\renewcommand{\chaptermark}[1]{%
	\markboth{\sl \chaptername~\thechapter.~#1}{}}
\renewcommand{\sectionmark}[1]{%
	\markright{\sl \thesection.~#1}}
  
	\fancyhf{}
	\lhead[\fancyplain{}{\leftmark}]{}
	\rhead[]{\fancyplain{}{\rightmark}}
	\lfoot[\thepage]{}
	\rfoot[]{\thepage}%*\label{line:lvtotnghiep_definition_rfoot}*)
	
	
% các định nghĩa sử dụng trong trang bìa
\makeatletter %*\label{line:lvtotnghiep_definition_makeatletter}*)
\newcommand{\@khoa}{Khoa gì?}
\newcommand{\khoa}[1]{\gdef\@khoa{#1}}
\newcommand{\@nganh}{Ngành học?}
\newcommand{\nganh}[1]{\gdef\@nganh{#1}}

\newcommand{\@chuongtrinhdaotao}{Chương trình gì?}
\newcommand{\chuongtrinhdaotao}[1]{\gdef\@chuongtrinhdaotao{#1}}

\newcommand{\@nam}{Năm ?}
\newcommand{\nam}[1]{\gdef\@nam{#1}}

\newcommand{\@thayhuongdan}{Tên thầy hướng dẫn ?}
\newcommand{\thayhuongdan}[1]{\gdef\@thayhuongdan{#1}} %*\label{line:lvtotnghiep_definition_thayhuongdan}*)

% trang bìa
\renewcommand{\maketitle}{ %*\label{line:lvtotnghiep_definition_maketitle}*)
\begin{titlepage}%

\newgeometry{top=2cm, bottom=2cm, left=2cm, right=2cm}%*\label{line:lvtotnghiep_definition_maketitle_geo}*)
	
\AddToShipoutPicture*{%	
\AtTextCenter{%
	\makebox(0,0)[c]{
	\begin{tikzpicture}
 	\draw[line width = 4pt](0,0) rectangle (\textwidth+2*0.3cm, 
 		\textheight+2*0.3cm);
	\end{tikzpicture}
 	}
}
}

\begin{center}
	
{\fontsize{13}{15.6}\selectfont 
 ĐẠI HỌC QUỐC GIA HÀ NỘI\\
 ĐẠI HỌC KHOA HỌC TỰ NHIÊN\\
 \textbf{KHOA: \MakeUppercase\@khoa}} 

	\vfill
	\textbf{\@author}
	\vfill
 	{\fontsize{18}{21.6}\selectfont \textbf{\MakeUppercase\@title}}
 	\vfill
 	
	 Khóa luận tốt nghiệp đại học hệ chính quy
	\vspace{\baselineskip}
 	
	Ngành: \@nganh\\
	(Chương trình đào tạo: \@chuongtrinhdaotao)
    
    \vfill
    \vfill
    
     \textbf{Hà Nội - \@nam}
  
  \end{center}
 \end{titlepage}%
} %*\label{line:lvtotnghiep_definition_maketitle_end}*)


% trang bìa phụ
\newcommand{\maketitleTrangphu}{%*\label{line:lvtotnghiep_definition_maketitleTrangphu}*)
\begin{titlepage}%

\AddToShipoutPicture*{%	
\AtTextCenter{%
	\makebox(0,0)[c]{
	\begin{tikzpicture}
	\draw[line width = 4pt](0,0)rectangle(\textwidth+2*0.3cm,
		\textheight+2*0.3cm);
 	\end{tikzpicture}
 	}
 }
}


\begin{center}
		
 {\fontsize{13}{15.6}\selectfont% 
 ĐẠI HỌC QUỐC GIA HÀ NỘI\\
 ĐẠI HỌC KHOA HỌC TỰ NHIÊN\\
 \textbf{KHOA: \MakeUppercase\@khoa}}

	\vfill
	\textbf{\@author}
	\vfill
 	{\fontsize{18}{21.6}\selectfont \textbf{\MakeUppercase\@title}}
	 \vfill
	Khóa luận tốt nghiệp đại học hệ chính quy
	\vspace{\baselineskip}
 	Ngành: \@nganh\\
    (Chương trình đào tạo: \@chuongtrinhdaotao)
    
    \vfill
	\textbf{Cán bộ hướng dẫn: \@thayhuongdan}
    \vfill
     
    \textbf{Hà Nội - \@nam}
    
\end{center}
\end{titlepage}%
}%*\label{line:lvtotnghiep_definition_maketitleTrangphu_end}*)

\makeatother

\title{Ứng dụng các phổ Gamma trong nghiên cứu cấu trúc hạt nhân 156GD}%*\label{line:lvthacsi_main_title}*)
\author{Đoàn Quang Tuyền}
\chuyennganh{Vật lý hạt nhân}
\maso{101.102.3}
\nam{2015}
\thayhuongdan{TS. Nguyễn Văn A}%*\label{line:lvthacsi_main_thayhuongdan}*)


% bắt đầu phần nội dung của luận văn
\begin{document}    

\pagestyle{empty}%*\label{line:lvthacsi_main_empty}*)
\lfoot[]{}
\rfoot[]{}

\maketitle

\cleardoublepage
\maketitleTrangphu

	\chapter*{Lời cam đoan}

Trình bày nội dung của phần Lời cam đoan. Nội dung của phần này thường được trình bày trong 1 trang.
%*\label{line:lvthacsi_main_loicamdoan}*)

\clearHeading
 \lfoot[\thepage]{}
 \rfoot[]{\thepage}
 \setcounter{page}{1}%*\label{line:lvthacsi_main_setcounter}*)

\makeatletter
\renewcommand{\cftmarktoc}{\markboth{\sl \@title}{\sl \contentsname}}%*\label{line:lvthacsi_main_marktoc}\xdyIndex{markboth@\spverb\markboth}\xdyIndex{cftmarktoc@\spverb\cftmarktoc}*)
\makeatother
\tableofcontents%*\label{line:lvthacsi_main_tableofcontents}*)
\clearHeading

\makenomenclature
\renewcommand{\nomname}{Danh mục các từ viết tắt và các ký hiệu toán học}
\makeatletter
	\markboth{\sl \@title}{\sl \nomname}%*\label{line:lvthacsi_main_nom_markboth}\xdyIndex{markboth@\spverb\markboth}*)
\makeatother
\printnomenclature
\clearHeading

	\chapter{Mở dầu}
Trình bày tổng quan về vấn đề, mục đích của báo cáo và các công việc được thực hiện và các kết quả chính cùng với các kết luận v.v. 

Nội dung của báo cáo: 

\begin{itemize}
\item Chương \ref{ch:cosolythuyet}: trình bày cơ sở lý thuyết
\item Chương \ref{ch:thucnghiem}: trình bày phương pháp thực nghiệm
\item Chương \ref{ch:ketquaphantich}: Các kết quả phân tích
\item Chương \ref{ch:ketluan}: Các kết luận chính
\end{itemize}
%*\label{line:lvthacsi_main_modau}*)
	\clearHeading
	 
	\input{texfile/tongquan}
	\clearHeading
	
	\chapter{Đối tượng và phương pháp nghiên cứu}
\label{ch:doituongnghiencuu}
\minitoc

%\newpage

\section{Một số loại detector}
\index{detector}

Trong Chương \ref{ch:tongquan} đã trình bày các nguyên lý ghi nhận bức xạ, trong phần này sẽ trình bày các phương pháp thực nghiệm để ghi nhận các tia $\gamma$

\begin{table}[!h]
\caption{So sánh một số detector}
\centering
\begin{tabular}{|c|c|c|c|}
\hline 
Loại Detector & Hệ thống điện tử & Thuật toán & sử dụng Trigger \\ 
\hline 
EUROBALL & analog & addback & có \\ 
\hline 
EXOGAM & analog & addback & có \\ 
\hline 
JUROGAM & analog + digital & không & không \\ 
\hline 
AGATA & digital & tracking $\gamma$ & trigger+triggerless \\ 
\hline 
\end{tabular} 
\label{table:detector}
\end{table}

Một detector ghi nhận $\gamma$ được thiết lập từ các detector nhỏ hơn trong đó mỗi detector nhỏ sẽ gồm các khối Germanium \nomenclature{Ge}{Germanium} được đặt trong một lớp bảo vệ. Các khối Germanium có thể được phân chia thành các vùng khác nhau  và mỗi vùng này sẽ được kết nối với một điện cực riêng. Trong phần này sẽ giới thiệu sơ lược về bốn loại detector khác nhau: EXOGAM, EUROBALL, JUROGAM và AGATA \cite{bib_Bazzaco, bib_Simpson} \nomenclature{AGATA}{ Advanced GAmma Tracking Array} với các thông số chính như trong Bảng \ref{table:detector}. Mỗi loại detector đều có những ưu điểm riêng, tuy nhiên detector AGATA cho phép ghi nhận các tia bức xạ  $\gamma$ với độ phân giải tốt nhất. Để có thể tái lập năng lượng của các tia bức xạ $\gamma$, thuật toán addback và tracking đã được sử dụng.

Các kết quả thực nghiệm sẽ được trình bày trong Chương \ref{ch:ketqua}.

\renewcommand\bibname{Tài liệu tham khảo chương \thechapter}
\bibliographystyle{ieeetr}
\bibliography{bibtex/biblio} %put path to the Bib from where the BibtexRun.bat is located 
	\clearHeading
		
	\input{texfile/ketquavathaoluan}	
	\clearHeading

	\chapter{Kết luận}
\label{ch:ketluan}
Đưa ra các kết luận về các kết quả thu được, đề xuất các phương pháp thực nghiệm hoặc các phương pháp xử lý số liệu nếu cần v.v.

Có thể đưa các đoạn văn bản, công thức toán học, hình vẽ, bảng số v.v. vào đây. Có thể bổ sung thêm các mục hoặc các tiểu mục khác thông qua lênh section, subsection hay subsubsection v.v.
	\clearHeading
	
\defbibheading{tiengViet}{ 
	\section*{Tài liệu tiếng Việt}
	\markboth{\sl Tài liệu tham khảo}{\sl Tài liệu tiếng Việt}
	\addcontentsline{toc}{section}{Tài liệu tiếng Việt}
}

\defbibheading{tiengAnh}{
	\section*{Tài liệu tiếng Anh}
	\markboth{\sl Tài liệu tham khảo}{\sl Tài liệu tiếng Anh}
	\addcontentsline{toc}{section}{Tài liệu tiếng Anh}
}

\printbibheading[heading=bibintoc, title = {Tài liệu tham khảo}] 
\printbibliography[heading=tiengViet, keyword=VN]
\printbibliography[heading=tiengAnh, keyword=EN]
\clearHeading

\renewcommand{\appendixtocname}{Phụ lục}
\addappheadtotoc
\renewcommand{\chaptername}{\appendixname}

\appendix
	\chapter{Phổ gamma từ các detector khác nhau}
\label{phuluc:cacdothi}

\section{Các phổ gamma từ detector EUROBALL} 
Trong phần này trình bày tất cả các kết quả thực nghiệm, các thí nghiệm, các đồ thị thu được trong quá trình làm thực nghiệm, v.v. Phần này được sử dụng để dẫn chiếu.

\section{Các phổ gamma từ detector JUROGAM}
Trong phần này trình bày tất cả các kết quả thực nghiệm, các thí nghiệm, các đồ thị thu được trong quá trình làm thực nghiệm, v.v. Phần này được sử dụng để dẫn chiếu.
	\clearHeading
	\chapter{Các số liệu thực nghiệm}
\label{phuluc:solieuthucnghiem}

\section{Số liệu từ detector EUROBALL} 
Trong phần này trình bày tất cả các kết quả thực nghiệm, các thí nghiệm, các đồ thị thu được trong quá trình làm thực nghiệm, v.v. Phần này được sử dụng để dẫn chiếu.

\section{Số liệu từ detector JUROGAM}
Trong phần này trình bày tất cả các kết quả thực nghiệm, các thí nghiệm, các đồ thị thu được trong quá trình làm thực nghiệm, v.v. Phần này được sử dụng để dẫn chiếu.
	\clearHeading
     
\end{document}